% Monitoramento de servidores Linux por web sites.
%====================================================================================================
% TCC
%----------------------------------------------------------------------------------------------------
% Autor				    : Eduardo Balan
% Orientador		  : Kleber Krugrer
% Instituição 		: UFMS - Universidade Federal do Mato Grosso do Sul
% Departamento		: CPCX - Sistema de Informação
%----------------------------------------------------------------------------------------------------
% Data de criação	: 29 de Março de 2017
%====================================================================================================

\chapter{Fundamentação Teórica}\label{cap:funTeorica}

Neste Capítulo são apresentados os conceitos utilizados neste trabalho de acordo com a literatura estudada. Na seção \ref{sec:ArquiteturaClienteServidor} explica-se os conceitos da arquitetura cliente-servidor. Na Seção \ref{sec:WebServices} explicasse os conceitos de um \textit{web-service}. Na Seção \ref{sec:BancodeDados} explica-se o conceito de Banco de Dados. Na Seção \ref{sec:Http} explica-se o conceito do protocolo HTTP/1.1 seus métodos forma de utilização e respostas esperadas. Na Seção \ref{sec:ArquiteturaReST} é apresentada o estilo de desenvolvimento da arquitetura ReST e na Seção \ref{sec:TestesAutomatizados} são mostradas as principais vantagens e melhorias adquiridas por meio dos testes automatizados.


\section{Arquitetura Cliente-Servidor}\label{sec:ArquiteturaClienteServidor}

		Em uma arquitetura cliente-servidor há um hospedeiro sempre em funcionamento, denominado servidor, que atende à requisições de muitos outros hospedeiros, denominados clientes. Estes podem estar em funcionamento esporadicamente ou a todo o tempo. Um exemplo clássico é a aplicação \textit{web} na qual um servidor \textit{web} que está sempre em funcionamento atende à requisições de \textit{browsers} de hospedeiros clientes. Ao receber uma requisição de um objeto de um hospedeiro cliente, o servidor \textit{web} responde enviando o objeto requisitado a ele. Observe que, na arquitetura cliente-servidor, os clientes não se comunicam diretamente uns com os outros: por exemplo, na aplicação \textit{web}, dois \textit{browsers} não se comunicam diretamente. Outra característica da arquitetura cliente-servidor é que o servidor tem um endereço fixo, bem conhecido, denominado endereço IP (\textit{Internet Protocol}). Devido a esse característica do servidor e devido ao fato de ele estar sempre em funcionamento, um cliente sempre pode contatá-lo enviando um pacote ao endereço do servidor. Algumas das aplicações mais conhecidas que empregam a arquitetura cliente-servidor são \textit{web}, FTP (\textit{File Transfer Protocol}), Telnet e e-mail. Essa arquitetura cliente-servidor é mostrada na \autoref{Img:ArquiteturaClienteServidor} em que diversos clientes utilizando, computadores, notebooks e celulares realizam comunicação com um servidor \cite{Kurose:2010}.

Em aplicações cliente-servidor, muitas vezes acontece de um único hospedeiro servidor ser incapaz de atender a todas as requisições de seus clientes. Um site \textit{web} pode ficar rapidamente saturado se tiver apenas um servidor para atender grande número de requisições. Por essa razão, um grande conjunto de hospedeiros - às vezes coletivamente chamados \textit{data center} - freqüentemente é usado para criar um servidor virtual poderoso em arquitetura cliente-servidor, aumentando a capacidade de resposta de requisições \cite{Kurose:2010}.


\begin{figure}
	\centering
	\includegraphics[width=0.5\textwidth]{figuras/KuroseClienteServidorpg88.jpg}
	\caption[Arquitetura Cliente-Servidor.]{Representa a arquitetura clitente-Servidor, onde diversos tipos de cliente se comunicam com um mesmo servidor.}
	\label{Img:ArquiteturaClienteServidor}
\end{figure}


\section{\textit{Web Services}}\label{sec:WebServices}


O \textit{Web Service} é um \textit{software}  que atende a uma função de negócio específica para seus clientes. Ele recebe requisições de clientes e as responde ocultando todo o detalhamento do seu processamento. Normalmente as transações estão em formato XML (\textit{Extended Markup Language}), e são transmitidas utilizando o protocolo HTTP (\textit{HyperText Transfer Protocol}), entretanto podem ser utilizados outros protocolos de transporte \cite{Sampaio:2003}.

Os \textit{Web Services}, aliados aos padrões estabelecidos da Internet, podem fazer com que os dados fluam entre as várias entidades, como o governo e empresas interagindo entre si, reduzindo custos de modo a possibilitar maior facilidade e rapidez na troca de informações \cite{Abinader:2006}. Um exemplo utilizado nacionalmente é a NF-e (Nota Fiscal Eletrônica) em que diversas empresas com \textit{softwares} diferentes enviam arquivos XML utilizando o protocolo https (\textit{Hyper Text Transfer Protocol Secure}) para o \textit{web service} da receita federal, aguardam a realização dos processamentos de validação e recebem um arquivo XML de retorno com as informações desejadas.

Um \textit{web wervice} deve executar unidades completas de trabalho, não dependendo do estado de outros componentes externos. Outra definição importante é que eles devem ser "\textit{Stateless}" ou seja, cada requisição é considerada uma transação independente que não está relacionada a qualquer requisição anterior de forma que a comunicação consista de pares de requisição e resposta independentes \cite{Sampaio:2003, W3C:2001}.


\section{HTTP/1.1 \textit{HyperText Transfer Protocol}} \label{sec:Http}

O protocolo HTTP (\textit{HyperText Transfer Protocol}) é o protocolo utilizado em toda a \textit{World Wide Web}. Ele especifica o formato das mensagens que os clientes enviam aos servidores e também o formato de respostas que receberão. Cada interação consiste em uma solicitação ASCII, seguida por uma resposta RFC 822. 

Na utilização do protocolo HTTP/1.1 todos os clientes e todos os servidores devem obedecer a esse protocolo que é definido na RFC 2616 \cite{Tanenbaum:2003}.

\subsection{Conexões} \label{HTTP/1.1 - HyperText Transfer Protocol}

O modo habitual de cliente entrar em contato com um servidor é estabelecer uma conexão TCP (\textit{Transmission Control Protocol}) para a porta 80 da máquina servidora, embora esse procedimento não seja exigido formalmente. A vantagem de se usar o TCP é que ambas as partes não têm de se preocupar com mensagens perdidas, mensagens duplicadas, mensagens longas ou confirmações. Todos esses assuntos são tratados pela implementação do TCP \cite{Tanenbaum:2003}.

O HTTP/1.1 diferentemente de seus antecessores, admite conexões persistentes. Com elas, é possível estabelecer uma conexão TCP, enviar uma solicitação e obter uma resposta, e depois enviar solicitações adicionais e receber respostas adicionais. Amortizando o custo da instalação e da liberação do TCP por várias solicitações, o processamento relativo devido ao TCP é muito menor por solicitação. Também é possível transportar as solicitações por \textit{pipeline}, ou seja, enviar a segunda solicitação antes de chegar a resposta da primeira \cite{Tanenbaum:2003}.



\subsection{Métodos} \label{subsec:Metodos}

O HTTP foi criado de modo mais geral que simplesmente para utilização na \textit{web}, visando às futuras aplicações orientadas a objetos. Por essa razão, são aceitas operações chamadas métodos, diferentes da simples solicitação de uma página da \textit{web}. Cada solicitação consiste em uma ou mais linhas de texto ASCII, sendo a primeira palavra da primeira linha o nome do método solicitado. Os métodos internos estão listados na \autoref{Tab:MetodosHTTP} \cite{Tanenbaum:2003}.


% Please add the following required packages to your document preamble:
% \usepackage[table,xcdraw]{xcolor}
% If you use beamer only pass "xcolor=table" option, i.e. \documentclass[xcolor=table]{beamer}
\begin{table}[!ht]
\centering
\begin{tabular}{|l|l|}
\hline
{\color[HTML]{000000} \textbf{Método}} & {\color[HTML]{000000} \textbf{Descrição}} 
\\ \hline
GET                                    & \multicolumn{1}{p{13.50cm}|}{O método GET solicita ao servidor que envie a página (ou objeto, no caso mais genérico; na prática,apenas um arquivo). A grande maioria das solicitações a servidores da \textit{web} tem a forma de métodos GET.} 
\\ \hline
HEAD                                   & \multicolumn{1}{p{13.50cm}|}{O método HEAD solicita apenas o cabeçalho da mensagem, sem a página propriamente dita. Essemétodo pode ser usado para se obter a data da última modificação feita na ágina, para reunirinformações destinadas à indexação, ou apenas para testar a validade de um URL(Uniform Resource Locator).}
\\ \hline
PUT                                    & \multicolumn{1}{p{13.50cm}|}{O método PUT grava em uma página já existe. Esse método possibilita a criação de um conjunto de páginas da \textit{web} em um servidor remoto. O corpo da solicitação contém a página. } 
\\ \hline
POST                                   & \multicolumn{1}{p{13.50cm}|}{O método POST grava novos dados e são "anexados" a ele, em um sentido mais genérico\cite{Tanenbaum:2003}.} 
\\ \hline
DELETE                                 & \multicolumn{1}{p{13.50cm}|}{O método DELETE exclui a página. Não há garantia de que DELETE tenha sido bem-sucedido pois, mesmo que oservidor HTTP remoto esteja pronto para excluir a página, o arquivo subjacente pode ter um modoque impeça o servidor HTTP de modificá-lo ou excluí-lo \cite{Tanenbaum:2003}. }
\\ \hline
TRACE                                  & \multicolumn{1}{p{13.50cm}|}{O método TRACE serve para depuração. ele instrui o servidor a enviar de volta a solicitação. Essemétodo é útil quando as solicitações não estão sendo processadas corretamente e o cliente desejasaber qual solicitação o servidor recebeu de fato \cite{Tanenbaum:2003}.}   
\\ \hline
CONNECT                                & \multicolumn{1}{p{13.50cm}|}{O método CONNECT não é usado atualmente. Ele é reservado para uso futuro \cite{Tanenbaum:2003}.} \\ \hline
OPTIONS                                & \multicolumn{1}{p{13.50cm}|}{O método OPTIONS fornece um meio para que o cliente consulte o servidor sobre suas propriedades ou sobre as de um arquivo específico \cite{Tanenbaum:2003}.}
\\ \hline
\end{tabular}
\caption[Métodos da solicitações HTTP.]{Métodos da solicitações HTTP e suas descrições \cite{Tanenbaum:2003}.}
\label{Tab:MetodosHTTP}
\end{table}





\subsection{Códigos de Resposta} \label{subsec:Resposta}

Toda solicitação obtém uma resposta que consiste em uma linha de status e, possivelmente,
informações adicionais (por exemplo, uma página da \textit{web} ou parte dela). A linha de status contém
um código de status de três dígitos informando se a solicitação foi atendida e, se não foi, o motivo. 
O primeiro dígito é usado para dividir as respostas em cinco grupos importantes, como mostra-se na \autoref{Tab:FalhasHTTP} \cite{Tanenbaum:2003}.


% Please add the following required packages to your document preamble:
% \usepackage[table,xcdraw]{xcolor}
% If you use beamer only pass "xcolor=table" option, i.e. \documentclass[xcolor=table]{beamer}
\begin{table}[!h]
\centering
%\caption{My caption}
\begin{tabular}{|l|l|}
\hline
%\multicolumn{1}{|c|}{{\color[HTML]{000000} \textbf{Código}}} & \multicolumn{1}{c|}{{\color[HTML]{000000} \textbf{Descrição}}}    \\ \hline
\textbf{Código}   & \textbf{Descrição}       																						 \\ \hline
1xx   & Raramente são usados na prática \cite{Tanenbaum:2003}.        																						 \\ \hline
2xx   & Solicitação foi tratada com sucesso \cite{Tanenbaum:2003}.            																		 \\ \hline
3xx   & \begin{tabular}[c]{@{}l@{}}Informam ao cliente que ele deve procurar em outro lugar, 
				\\ usando uma URL diferente \cite{Tanenbaum:2003}.\end{tabular} 																						\\ \hline
4xx   & \begin{tabular}[c]{@{}l@{}}Solicitação falhou devido a um erro do cliente,
				\\ como uma solicitação inválida ou uma página inexistente \cite{Tanenbaum:2003}.\end{tabular}             \\ \hline
5xx   & \begin{tabular}[c]{@{}l@{}}O próprio servidor tem um problema, seja causado por um  
				\\ erro em seu código ou por uma sobrecarga temporária \cite{Tanenbaum:2003}.\end{tabular}                  \\ \hline
\end{tabular}
\caption[Códigos de resposta da solicitação HTTP.]{Códigos de resposta da solicitação HTTP \cite{Tanenbaum:2003}.}
\label{Tab:FalhasHTTP}
\end{table}


\subsection{Exemplo de utilização do HTTP/1.1} \label{subsec:ExemploHttp}

Este protocolo foi desenvolvido de maneira a ser o mais flexível possível para comportar diversas necessidades diferentes.
Sempre que uma requisição é enviada ela segue o formato da requisições que pode ser visto no \autoref{Func:GETModelo} \cite{Saudate:2014}.

\begin{lstlisting}[label=Func:GETModelo,caption={[Formato de uma requisição HTTP]Formato de uma requisição HTTP}]
<método> <URL> HTTP/<versão>
<Cabeçalhos - Vários cabeçalhos podem ser enviados mas um em cada linha>

<corpo da requisição> 
\end{lstlisting}


Quando um GET para \url{http://www.site.com.br/clientes} é realizado uma requisição é criada como pode ser visto no \autoref{Func:GETExemplo}. Nesta requisição pode ser visto em sua primeira linha o método GET da solicitação seguido pela URL a qual está sendo feito a solicitação e pela versão do protocolo HTTP. O servidor responsável por receber a solicitação e a porta são passado no cabeçalho da requisição precedido pela palavra Host. Na terceira linha pode ser visto um cabeçalho \textit{Accept}, que serve para informar ao servidor qual tipo de dados a requisição espera receber \cite{Saudate:2014}.


\begin{lstlisting}[label=Func:GETExemplo,caption={[Exemplo de uma requisição HTTP utilizando o método GET.]Exemplo de uma solicitação HTTP utilizando o método GET, para \url{http://www.site.com.br/clientes} com o pedido de uma arquivo XML de resposta.}]
GET /clientes HTTP/1.1
Host: www.site.com.br:80
Accept: text/xml
\end{lstlisting}

Assim que o servidor receber a solicitação ele ira processar as informações, e retornara uma resposta com um dos código mostrado no \autoref{Tab:FalhasHTTP}. A resposta retornada possuirá o formato mostrado no \autoref{Func:RespostaModelo} \cite{Saudate:2014}.


\begin{lstlisting}[label=Func:RespostaModelo,caption={[Formato de uma resposta HTTP]Formato de uma resposta HTTP}]
HTTP/<versão> <código de status> <descrição do código>
<cabeçalhos>
<resposta>
\end{lstlisting}


Para a solicitação mostrada no \autoref{Func:GETExemplo} uma resposta velida seria a representada pelo \autoref{Func:RespostaExemplo}.
Nesta resposta pode ser visto em sua primeira linha a versão do protocolo HTTP. seguido pelo código de resposta e descrição do código.
Na segunda linha temos um cabeçalho Content-Type que indica qual o formato da resposta, através de um tipo conhecido como \textit{Media Type}.
A terceira linha tem um cabeçalho Content-Length que informa qual o tamanho da resposta.
A partir da quarta linha temos a resposta em XML da solicitação \cite{Saudate:2014}.



\begin{lstlisting}[label=Func:RespostaExemplo,caption={[Exemplo de uma resposta HTTP com status 200.]Exemplo de uma resposta HTTP com status 200, possuindo um Content-Type:text/xml informando que a resposta é em formato XML, e a resposta a solicitação.}]
HTTP/1.1 200 OK
Content-Type: text/xml
Content-Length: 232
<clientes>
	<cliente>
		<id>1</id>
		<nome>Alexandre</nome>
		<dataNascimento>2012-12-01</dataNascimento>
	</cliente>
	<cliente>
		<id>2</id>
		<nome>Paulo</nome>
		<dataNascimento>2012-11-01</dataNascimento>
	</cliente>
</clientes>
\end{lstlisting}


\section{Arquitetura ReST} \label{sec:ArquiteturaReST}

Os princípios básicos de ReST (\textit{Representational State Transfer}) são estilos de desenvolvimento
de \textit{web services} que teve origem na tese de doutorado de Roy Fielding. 
Este, por sua vez, é co-autor de um dos protocolos mais utilizados no mundo, o HTTP/1.1 (\textit{HyperText Transfer Protocol}) \cite{Boagrio:2017} \cite{Saudate:2014}.  
Assim, é notável que o protocolo ReST é guiado pelo que seriam as boas práticas de uso do HTTP/1.1 \cite{Saudate:2014}:


Em ReST, cada recurso deve ter uma URL bem definida. Por exemplo, o conjunto dos usuários de um sistema pode ter mapeada para si uma URL \url{http://www.site.com/usuarios}. Caso queiramos apenas o usuário de ID 1, essa URL se torna \url{http://www.site.com/usuarios/1} \cite{Saudate:2012}.

Parâmetros adicionais que não fazem parte da definição do recurso propriamente dito e/ou sejam opcionais podem ser passados em formato de \textit{query string}, ou seja, dados “anexados” à URL. 
Esses dados devem ser passados após o '?' e contendo um nome, seguido de '=' e seu respectivo valor, outros dados adicionais devem ser separados por '\&'. 
Por exemplo, para utilizar paginação em um listagem de usuários, deve ser passado um \textit{query string} contendo os valores desejados da paginação, como na requisição \url{http://www.site.com/usuarios?pagina=1\&itemPorPagina=20} \cite{Saudate:2012}.

Note que as URLs seguem uma estrutura hierárquica, ou seja, o elemento seguinte obedece a um relacionamento com o elemento anterior. Ou seja, se quisermos o endereço do usuário, devemos obter primeiro o usuário em questão e, depois, o endereço. Assim sendo, a URL seria \url{http://www.site.com/usuarios/1/endereco} . No entanto, se quiséssemos obter apenas o endereço de todos os usuários, aí teríamos uma URL \url{http://www.site.com/usuarios/endereco}. O endereço obedece a uma estrutura hierárquica em relação a usuários \cite{Saudate:2012}.


\section{Testes Automatizados} \label{sec:TestesAutomatizados}

Todo desenvolvedor de software já escreveu um trecho de código que não funcionava. E muitas vezes só descobriu que o código não funciona quando o cliente reporta o \textit{bug}. Nesse momento o desenvolvedor perde confiança no código, e seu cliente perde a confiança na equipe de desenvolvimento \cite{Aniche:2015}.

Uma maneira para conseguir testar o sistema todo de maneira constante e contínua é automatizando os testes. Ou seja, escrevendo um programa que testa o seu programa. Esse programa invocaria os comportamentos do seu sistema e garantiria que a saída é sempre a esperada. Se isso for feito, teríamos diversas vantagens. O teste executaria muito rápido (afinal, é uma máquina!). Se ele executa rápido, logo ele seria rodado constantemente. Se for rodado constantemente, os problemas serão encontrados mais cedo. \cite{Aniche:2012}.

\subsection{Testes de Unidade}\label{Testes Automatizados}

Um teste de unidade não se preocupa com todo o sistema; ele está interessado apenas em saber se uma pequena parte do sistema funciona. Ele testa uma única unidade do nosso sistema. Geralmente, em sistemas orientados a objetos, essa unidade é a classe.

Testes automatizados são fundamentais para um desenvolvimento de qualidade. Sua existência traz diversos benefícios para o software, como o aumento da qualidade e a diminuição de \textit{bugs} em produção \cite{Aniche:2012}


