% Monitoramento de servidores Linux por web sites.
%====================================================================================================
% TCC
%----------------------------------------------------------------------------------------------------
% Autor				    : Eduardo Balan
% Orientador		  : Kleber Krugrer
% Instituição 		: UFMS - Universidade Federal do Mato Grosso do Sul
% Departamento		: CPCX - Sistema de Informação
%----------------------------------------------------------------------------------------------------
% Data de criação	: 29 de Março de 2017
%====================================================================================================

\chapter{Anexos} \label{App:ApendiceA}


\begin{table}[!ht]
\centering
\begin{tabular}{|l|l|}
\hline
{\color[HTML]{000000} \textbf{Variável}} & {\color[HTML]{000000} \textbf{Descrição}}                                      \\ \hline
id                                       & Número único para cada objeto do tipo ServidorConfig.\\ 
																				 &(Gerado Automaticamente) 																												\\ \hline
servidor                                 & Indica com qual objeto Servidor esse registro esta relacionado.                       \\ \hline
dthr\_cadastro                           & Data e hora do cadastro. (Gerado Automaticamente)                              \\ \hline
nome                                     & Nome do processador.                                                           \\ \hline
cacheSize                                & Tamanho do cache do processador.                                               \\ \hline
cpuCores                                 & Quantos núcleos tem o processador.                                             \\ \hline
siblings                                 & Quantos núcleos virtuais tem o processador.                                    \\ \hline
\end{tabular}
\caption[Variáveis da classe InformacoesCpu e suas descrições.]{Variáveis da classe InformacoesCpu e suas descrições.}
\label{Tab:VariaveisInformacoesCpu}
\end{table}



\begin{table}[!ht]
\centering
\begin{tabular}{|l|l|}
\hline
{\color[HTML]{000000} \textbf{Variável}} & {\color[HTML]{000000} \textbf{Descrição}}                                      \\ \hline
id                                       & Número único para cada objeto do tipo ServidorConfig.\\ 
																				 &(Gerado Automaticamente) 																												\\ \hline
servidor                                 & Indica com qual objeto Servidor esse registro esta relacionado.                       \\ \hline
dthr\_cadastro                           & Data e hora do cadastro. (Gerado Automaticamente)                              \\ \hline
total                                    & Quanto de memoria existe no servidor.                                          \\ \hline
\end{tabular}
\caption[Variáveis da classe InformacoesMemoria e suas descrições.]{Variáveis da classe InformacoesMemoria e suas descrições.}
\label{Tab:VariaveisInformacoesMemoria}
\end{table}


\begin{table}[!ht]
\centering
\begin{tabular}{|l|l|}
\hline
{\color[HTML]{000000} \textbf{Variável}} & {\color[HTML]{000000} \textbf{Descrição}}                                      \\ \hline
id                                       & Número único para cada objeto do tipo ServidorConfig.\\ 
																				 &(Gerado Automaticamente) 																												\\ \hline
servidor                                 & Indica com qual objeto Servidor esse registro esta relacionado.                       \\ \hline
dthr\_cadastro                           & Data e hora do cadastro. (Gerado Automaticamente)                              \\ \hline
total                                    & Quanto de memoria swap existe no servidor.                                     \\ \hline
\end{tabular}
\caption[Variáveis da classe InformacoesSwap e suas descrições.]{Variáveis da classe InformacoesSwap e suas descrições.}
\label{Tab:VariaveisInformacoesSwap}
\end{table}


\begin{table}[!ht]
\centering
\begin{tabular}{|l|l|}
\hline
{\color[HTML]{000000} \textbf{Variável}} & {\color[HTML]{000000} \textbf{Descrição}}                                      \\ \hline
id                                       & Número único para cada objeto do tipo ServidorConfig.\\ 
																				 &(Gerado Automaticamente) 																												\\ \hline
informacoesCpu                           & Indica com qual objeto InformacoesCpu esse registro \\ 
																				 &esta relacionado.          \\ \hline
dthr\_cadastro                           & Data e hora do cadastro. (Gerado Automaticamente)                              \\ \hline
coreId                                   & Id do núcleo que esta sendo monitorado.                                        \\ \hline
cpuMhz                                   & Quantos Mhz (MegaHertz) esta sendo utilizado.                                  \\ \hline
\end{tabular}
\caption[Variáveis da classe MonitoramentoCpu e suas descrições.]{Variáveis da classe MonitoramentoCpu e suas descrições.}
\label{Tab:VariaveisMonitoramentoCpu}
\end{table}

\begin{table}[!ht]
\centering
\begin{tabular}{|l|l|}
\hline
{\color[HTML]{000000} \textbf{Variável}} & {\color[HTML]{000000} \textbf{Descrição}}                                                                             \\ \hline
id                                       & \multicolumn{1}{p{10.00cm}|}{Número único para cada objeto do tipo ServidorConfig. (Gerado Automaticamente)}                                        \\ \hline
informacoesMemoria                       & \multicolumn{1}{p{10.00cm}|}{Indica com qual objeto InformacoesMemoria esse registro esta relacionado. }\\ \hline
dthr\_cadastro                           & \multicolumn{1}{p{10.00cm}|}{Data e hora do cadastro. (Gerado Automaticamente)} \\ \hline
active                                   & \multicolumn{1}{p{10.00cm}|}{A quantidade total de buffer (Memoria Temporaria) ou memória cache que foi utilizada recentemente e não foi liberada.} \\ \hline
memfree                                  & \multicolumn{1}{p{10.00cm}|}{A quantidade de  memoria física não utilizada.} \\ \hline
availabre                                & \multicolumn{1}{p{10.00cm}|}{A quantidade de memoria que pode ser acessada pelo servidor.} \\ \hline
buffers                                  & \multicolumn{1}{p{10.00cm}|}{A quantidade de memoria física utilizada para buffers  de arquivos.}   \\ \hline
\end{tabular}
\caption[Variáveis da classe MonitoramentoMemoria e suas descrições.]{Variáveis da classe MonitoramentoMemoria e suas descrições.}
\label{Tab:VariaveisMonitoramentoMemoria}
\end{table}

\begin{table}[!ht]
\centering
\begin{tabular}{|l|l|}
\hline
{\color[HTML]{000000} \textbf{Variável}} & {\color[HTML]{000000} \textbf{Descrição}}                                      \\ \hline
id                                       & \multicolumn{1}{p{10.00cm}|}{Número único para cada objeto do tipo ServidorConfig. (Gerado Automaticamente)}\\ \hline
InformacoesSwap                          & Indica com qual objeto InformacoesSwap esse registro esta  \\
																				 & relacionado.         \\ \hline
dthr\_cadastro                           & Data e hora do cadastro. (Gerado Automaticamente)                              \\ \hline
free                                     & O montante total de swap livre.                                                \\ \hline
cached                                   & A quantidade de swap que esta sendo utilizada.                                 \\ \hline
\end{tabular}
\caption[Variáveis da classe MonitoramentoSwap e suas descrições.]{Variáveis da classe MonitoramentoSwap e suas descrições.}
\label{Tab:VariaveisMonitoramentoSwap}
\end{table}


\begin{table}[!ht]
\centering
\begin{tabular}{|l|l|}
\hline
{\color[HTML]{000000} \textbf{Variável}} & {\color[HTML]{000000} \textbf{Descrição}}\\ \hline
id                                       &  \multicolumn{1}{p{10.00cm}|}{Número único para cada objeto do tipo ServidorConfigDb. (Gerado Automaticamente)} \\ \hline
servidorConfigDb                         &  \multicolumn{1}{p{10.00cm}|}{Indica com qual objeto ServidorConfigDb esse registro esta relacionado.} \\ \hline
dthr\_cadastro                           &  \multicolumn{1}{p{10.00cm}|}{Data e hora do cadastro. (Gerado Automaticamente)} \\ \hline
tipoExecucao                             &  \multicolumn{1}{p{10.00cm}|}{É uma enum do tipo EnumSgdbTipoExec o qual informa se o procedimento é um backup ou um vacuum. }\\ \hline
exito                                    &  \multicolumn{1}{p{10.00cm}|}{Informa se a procedimento foi realizado com sucesso. }\\ \hline
mensagem                                 &  \multicolumn{1}{p{10.00cm}|}{Mensagem gerada pelo procedimento. }\\ \hline
\end{tabular}
\caption[Variáveis da classe MonitoramentoPostgres e suas descrições.]{Variáveis da classe MonitoramentoPostgres e suas descrições.}
\label{Tab:VariaveisMonitoramentoPostgres}
\end{table}

\begin{table}[!ht]
\centering
\begin{tabular}{|l|l|}
\hline
{\color[HTML]{000000} \textbf{Variável}} & {\color[HTML]{000000} \textbf{Descrição}} \\ \hline
id                                       & \multicolumn{1}{p{10.00cm}|}{Número único para cada objeto do tipo ServidorConfigDb. (Gerado Automaticamente)}\\ \hline
ServidorConfigInformacoesDb              & \multicolumn{1}{p{10.00cm}|}{Indica com qual objeto ServidorConfigInformacoesDb esse registro esta relacionado. }\\ \hline
dthr\_cadastro                           & \multicolumn{1}{p{10.00cm}|}{Data e hora do cadastro. (Gerado Automaticamente)  }\\ \hline
listenAddresses                          & \multicolumn{1}{p{10.00cm}|}{De qual ip o postgreSQL ira aceitar conexão.}\\ \hline
port                                     & \multicolumn{1}{p{10.00cm}|}{Porta que o postgreSQL esta rodando.}\\ \hline
maxConnections                           & \multicolumn{1}{p{10.00cm}|}{Numero máximo de conexão que o postgresSQL ira aceitar simultaneamente. }\\ \hline
ssl                                      & \multicolumn{1}{p{10.00cm}|}{Se a conexão com o banco possui ssl (Secure Socket Layer). }\\ \hline
sharedBuffers                            & \multicolumn{1}{p{10.00cm}|}{Quantidade de memória compartilhada utilizada pelo postgreSQL.}\\ \hline
tempBuffers                              & \multicolumn{1}{p{10.00cm}|}{Quantidade de memória utilizada por sessão do postgresSQL.}\\ \hline
workMem                                  & \multicolumn{1}{p{10.00cm}|}{Quantidade de memória utilizada por operações de classificação interna e tabelas de hash antes de gravar em arquivos de disco.  }\\ \hline
maintenanceWorkMem                       & \multicolumn{1}{p{10.00cm}|}{Quantidade máxima de memória a ser usada por operações de manutenção, como VACUUM , CREATE INDEX e ALTER TABLE ADD FOREIGN KEY. }\\ \hline
maxStackDepth                            & \multicolumn{1}{p{10.00cm}|}{Especifica a profundidade máxima da pilha de execução do postgresSQL no servidor.}\\ \hline
maxPreparedTransactions                  & \multicolumn{1}{p{10.00cm}|}{Define o número máximo de transações que podem estar no estado "prepared".} \\ \hline
\end{tabular}
\caption[Variáveis da classe ServidorConfigInformacoesDb e suas descrições.]{Variáveis da classe ServidorConfigInformacoesDb e suas descrições.}
\label{Tab:VariaveisMonitoramentoPostgres}
\end{table}


\begin{table}
\centering
\begin{tabular}{|l|l|}
\hline
{\color[HTML]{000000} \textbf{Variável}} & {\color[HTML]{000000} \textbf{Descrição}}                                        \\ \hline
id                                       & Número único para cada objeto do tipo ServidorConfigDb. \\
																				 & (Gerado Automaticamente) 																												\\ \hline
dthr\_cadastro                           & Data e hora do cadastro. (Gerado Automaticamente)                                \\ \hline
nome                                     & Nome do usuário.                                                                 \\ \hline
sobrenome                                & Sobrenome do usuário.                                                            \\ \hline
login                                    & Login do usuário.                                                                \\ \hline
email                                    & Email do usuário.                                                                \\ \hline
senha                                    & Senha de acesso do usuário.                                                      \\ \hline
sexo                                     & Enum do tipo EnumSexo para informar o sexo do usuário.                           \\ \hline
telefone                                 & Telefone para contato com o usuário.                                             \\ \hline
\end{tabular}
\caption[Variáveis da classe Usuario e suas descrições.]{Variáveis da classe Usuario e suas descrições.}
\label{Tab:VariaveisUsuario}
\end{table}


\begin{table}[!ht]
\centering
\begin{tabular}{|l|l|}
\hline
{\color[HTML]{000000} \textbf{Variável}} & {\color[HTML]{000000} \textbf{Descrição}}                                        \\ \hline
id                                       & Número único para cada objeto do tipo ServidorConfigDb.\\
																				 & (Gerado Automaticamente) \\ \hline
dthr\_cadastro                           & Data e hora do cadastro. (Gerado Automaticamente)                                \\ \hline
nome                                     & Nome do Dominio.                                                                 \\ \hline
observacao                               & Observação para o dominio.                                                       \\ \hline
\end{tabular}
\caption[Variáveis da classe Dominio e suas descrições.]{Variáveis da classe Dominio e suas descrições.}
\label{Tab:VariaveisDominio}
\end{table}


\chapter{Anexos} \label{App:ApendiceB}

\begin{lstlisting}[style=Java, label=Func:GenericEntity,caption={[Entidade genérica da aplicação GenericEntity.]Entidade genérica da aplicação GenericEntity e suas funcionalidades.}]
package br.com.webmonitor.entity.Generic;

import java.io.Serializable;
import java.util.Date;
import java.util.Objects;

/**
 * Classe base para qualquer objeto serializável.
 *
 * @author Eduardo Balan
 *
 * @param <T> o tipo do atributo id
 */
public abstract class GenericEntity<T extends Serializable> implements Serializable {

    private static final long serialVersionUID = 1L;

    public abstract T getId();

    public abstract void setId(T id);

    public abstract Date getDthr_cadastro();

    public abstract void setDthr_cadastro(Date date);

    /**
     * Indica quando outro objeto é igual a este. Nesta implementação, qualquer objeto derivado de Bean é igual a este desde que seja exatamente da mesma classe e tenha o mesmo ID.
     *
     * @author Kleber Kruger
     *
     * @param obj o objeto a comparar com este
     * @return {@code true} se este objeto é igual ao do argumento; {@code false} caso contrário.
     */
    @Override
    public boolean equals(Object obj) {
        if (getId() != null && obj instanceof GenericEntity) {
            GenericEntity x = (GenericEntity) obj;
            return getClass() == x.getClass() && getId().equals(x.getId());
        }
        return super.equals(obj);
    }

    /**
     * Retorna um valor de hash code para este objeto. Nesta implementação, este valor é gerado por
     * uma combinação do hash code da classe (getClass().hashCode()) somado ao hash code do atributo
     * id (id.hashCode()).
     *
     * @author Kleber Kruger
     *
     * @return um valor de hash code para este objeto
     */
    @Override
    public int hashCode() {
        if (getId() != null) {
            return 43 * 7 + Objects.hashCode(getClass().hashCode() + getId().hashCode());
        }
        return super.hashCode();
    }

}
\end{lstlisting}

\begin{lstlisting}[style=Java, label=Func:GenericBO,caption={[Entidade genérica GenericBO.]Entidade genérica GenericBO e suas funcionalidades.}]
package br.com.webmonitor.business.generic;

import br.com.webmonitor.entity.Generic.GenericEntity;
import br.com.webmonitor.exception.SqlGenericRuntimeException;
import br.com.webmonitor.exception.SqlInexistenteRuntimeException;
import org.springframework.beans.factory.annotation.Autowired;
import org.springframework.data.jpa.repository.JpaRepository;


import javax.persistence.MappedSuperclass;
import java.util.Date;


/**
 * Class GenericBO é a classe responsável  por regras de negocio, genéricos e simples como:
 * Inserção em uma base de dados, remoção da base de dados, e update.
 *
 * @author Eduardo Balan
 *
 * @param Entity Entidade a qual ela ira prestar o servico.
 * @param Repository Repositorio responsavel pela Entity que vc esta utilizando.
 *
 * @throws SqlInexistenteRuntimeException
 * @throws SqlGenericRuntimeException
 */
@MappedSuperclass
public class GenericBO <Entity extends GenericEntity, Repository extends JpaRepository<Entity, Long>> {

    /* Repositorio responsavel pela Entity */
    @Autowired
    private Repository repository;

    /**
     * Metodo responsável pelas regras de negocio genéricas da inserção.
     *
     * @author Eduardo Balan
     *
     * @param Entity Entidade que será persistida no banco de dados.
     *
     * @throws SqlGenericRuntimeException
     *
     * return Entity persistida no banco de dados.
     */
    public Entity inserir(Entity entityNova){
        try{
            entityNova.setDthr_cadastro(new Date());
            return repository.save(entityNova);
        }catch (Exception e){
            throw new SqlGenericRuntimeException(e);
        }
    }

    /**
     * Metodo responsável pelas regras de negocio genéricas da exclusão.
     *
     * @author Eduardo Balan
     *
     * @param Long id da entidade que será removida do banco de dados.
     *
     * @throws SqlInexistenteRuntimeException
     * @throws SqlGenericRuntimeException
     *
     * return void.
     */
    public void excluir(Long idEntity){
        Entity entityPersistidaNoDB = repository.findOne(idEntity);
        if(entityPersistidaNoDB == null){
            throw new SqlInexistenteRuntimeException("Registro não encontrado na base de dados.", null);
        }
        try{
            repository.delete(idEntity);
        }catch (Exception e){
            throw new SqlGenericRuntimeException(e);
        }
    }

}

}
\end{lstlisting}
%As tabelas deste apêndice mostram os resultados individuais dos testes sem a biblioteca \textit{FaultRecovery}, com a biblioteca \textit{FaultRecovery}, sem a classe \textit{TData} e com a classe \textit{TData}.

%\includepdf[pages={1-17}]{anexos.pdf}
