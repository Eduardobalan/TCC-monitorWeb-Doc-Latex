%====================================================================================================
% Monitoramento de servidores Linux por web sites.
%====================================================================================================
% Plano de Trabalho
%----------------------------------------------------------------------------------------------------
% Autor					: Eduardo Balan
% Orientador		: Kleber Kruger
% Instituição 	: UFMS - Universidade Federal do Mato Grosso do Sul
% Unidade				: CPCX - Campus de Coxim
%----------------------------------------------------------------------------------------------------
% Arquivo			: plano_trabalho.tex
% Data de criação	: 29 de Março de 2017
%=====================================================================================================

\chapter*{Abstract}

The web has been increasingly used as a way to run applications. There are currently word processors, spreadsheets and other programs running as a web application, where your main information is stored on the servers \cite{Marimoto:2011}. It's easy to realize that a server crash can compromise user productivity, mainly if that server is the only one available or running vital services \cite{Weber:2002}.

The main objective of this work was to study about technologies such as Linux, C ++, Java, Spring and PostgreSQL to create a monitoring system for Linux servers using the client-server architecture. Another goal was to automatically perform backup and vacuum procedures on the PostgreSQL database. These procedures are of great importance to identify problems, once these servers can store vital data such as financial data, user accounts, corporate reports, software and hardware projects, among others.

In the implementation part, it was developed a system called MonitorWeb, consisting of two applications: the first, a client application called MonitorWeb-Cli, which performs monitoring of CPU, memory and swap at intervals defined according to the user, besides allow backup and vacuum procedures to be performed in sporadic times. The second consists of a web service that receives the monitoring data from the client applications and provides the informations through the ReST standard so that other applications can use them.

Finally, the results are described, to which the client application was submitted 15 times (10 times for 1-second interval monitoring and 5 for no intervals) to the Valgrind tool, in order to identify problems of memory leakage and inefficient allocation of resources. In the end, no problem was detected. The server application was subjected to a load of 86 unit tests, ensuring that all the resources work with greater reliability. Two test batteries were also runned: the first with common settings to a real environment (interval monitoring of 1-second) to check the performance. The second defined a configuration without interval between the monitoring, and was performed in order to check the performance of the server when exposed to an excessive load of requests per second. Both tests presented satisfactory results, having the second an average insertion rate of 546.34 records per second.