%====================================================================================================
% Monitoramento de servidores Linux por web sites.
%====================================================================================================
% Plano de Trabalho
%----------------------------------------------------------------------------------------------------
% Autor					: Eduardo Balan
% Orientador		: Kleber Kruger
% Instituição 	: UFMS - Universidade Federal do Mato Grosso do Sul
% Unidade				: CPCX - Campus de Coxim
%----------------------------------------------------------------------------------------------------
% Arquivo			: plano_trabalho.tex
% Data de criação	: 29 de Março de 2017
%=====================================================================================================

\chapter*{Resumo}

A \textit{web} vem sendo utilizada cada vez mais como uma forma de executar aplicativos. Atualmente existem processadores de texto, planilhas e outros programas sendo executados como uma aplicação \textit{web}, em que suas principais informações ficam armazenadas nos servidores \cite{Marimoto:2011}. É fácil perceber que a queda de um servidor pode comprometer a produtividade de usuários, principalmente se esse servidor for o único disponível ou se estiver executando serviços vitais \cite{Weber:2002}.

O objetivo deste trabalho foi estudar tecnologias como Linux, C++, Java, Spring e PostgreSQL para criar um sistema de monitoramento para servidores Linux utilizando a arquitetura cliente-servidor. Outro objetivo foi realizar automaticamente procedimentos de \textit{backup} e \textit{vacuum} no banco de dados PostgreSQL. Esses procedimentos são de grande importância para identificar problemas, uma vez que esses servidores podem armazenar dados vitais, tais como dados financeiros, contas de usuário, relatórios corporativos, projetos de \textit{softwares} e \textit{hardware}, entre outros.

Na parte de implementação, desenvolveu-se um sistema chamado MonitorWeb, composto por duas aplicações: a primeira, uma aplicação cliente denominada MonitorWeb-Cli, que realiza o monitoramento de CPU, memória e \textit{swap} em intervalos definidos de acordo com o usuário, além de permitir a realização de procedimentos de \textit{backup} e \textit{vacuum} em tempos esporádicos. A segunda, consiste em uma aplicação servidor (\textit{web service}) que recebe os dados de monitoramento das aplicações cliente e disponibiliza as informações por meio do padrão ReST para que outras aplicações possam utiliza-las.

Ao final são descritos os resultados, ao qual a aplicação cliente foi submetida 15 vezes (10 vezes fazendo monitoramento em intervalos de 1 segundo e outras 5 sem intervalos) à ferramenta Valgrind, com o intuito de identificar problemas de vazamento de memória e alocação ineficiente de recursos. Ao final, nenhum problema foi detectado. A aplicação servidor foi submetida a uma carga de 86 testes de unidade, garantindo que todos os recursos funcionem com maior confiabilidade. Também foi realizado duas baterias de teste: a primeira com configurações comum a um ambiente real (monitoramentos em intervalos de 1 segundo) para verificar o desempenho. A segunda definiu uma configuração sem intervalo entre os monitoramentos, e foi realizada com o intuito de verificar o desempenho do servidor quando exposto a uma carga excessiva de requisições por segundo. Ambos os testes apresentaram resultados satisfatórios, tendo o segundo uma taxa de inserção média de 546,34 registros por segundo.
